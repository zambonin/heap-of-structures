\documentclass{article}

\usepackage[brazil]{babel}
\usepackage[T1]{fontenc}
\usepackage[a4paper, margin=1.5cm]{geometry}
\usepackage[colorlinks, urlcolor=blue, citecolor=red]{hyperref}
\usepackage[utf8]{inputenc}
\usepackage{color, enumitem, listings}

\definecolor{codegreen}{rgb}{0,0.6,0}
\definecolor{codegray}{rgb}{0.5,0.5,0.5}
\definecolor{codepurple}{rgb}{0.58,0,0.82}
\definecolor{backcolour}{rgb}{0.95,0.95,0.92}

\lstdefinestyle{mystyle}{
    commentstyle=\color{codegreen},
    keywordstyle=\color{blue},
    numberstyle=\tiny\color{codegray},
    stringstyle=\color{codepurple},
    basicstyle=\footnotesize\ttfamily,
    breaklines=true,
    numbers=left,
    showstringspaces=false
}

\lstset{style=mystyle}

\title{\textbf{Complexidade de algoritmos: cálculo de Floyd}}
\author{
    Gustavo Zambonin, Lucas Ribeiro Neis\thanks{
        \texttt{\{gustavo.zambonin,lucas.neis\}@grad.ufsc.br} \hfill
        \texttt{\href{https://github.com/zambonin/ufsc-ine5408}{src/}}
    } \\
    \small{Estruturas de Dados (UFSC -- INE5408)}
}
\date{}

\begin{document}

\maketitle

\lstinputlisting[language=C++]{snippet.cpp}

De acordo com a proposta do trabalho, a análise do pseudocódigo acima será
realizada passo a passo, descrevendo os custos de cada linha detalhadamente.

\begin{enumerate}[label=Linha \arabic*:]

    \item Inicialização da variável \texttt{i}, comparação à variável
        \texttt{n}, incremento da variável \texttt{i} ($1$ unidade de custo
        para a inicialização, $n + 1$ para os testes de comparação e $n$
        para os incrementos): $\mathbf{2n + 2}$

    \item Análogo à linha 1, porém multiplicado pelo número de iterações
        do laço superior: $\mathbf{2n^2 + 2n}$

    \item Atribuição a uma variável ($1$ unidade de custo para a atribuição,
        além do número de iterações dos laços superiores, que multiplica o
        resultado final por $n^2$): $\mathbf{n^2}$

    \item Análogo à linha 3: $\mathbf{n^2}$ \addtocounter{enumi}{3}

    \item Análogo à linha 1: $\mathbf{2n + 2}$

    \item Atribuição a uma variável ($1$ unidade de custo para a atribuição,
        além do número de iterações do laço superior, que multiplica o
        resultado final por $n$): $\mathbf{n}$ \addtocounter{enumi}{2}

    \item Análogo à linha 1: $\mathbf{2n + 2}$

    \item Análogo à linha 2: $\mathbf{2n^2 + 2n}$

    \item Análogo à linha 2, porém multiplicado pelo número de iterações
        dos laços superiores: $\mathbf{2n^3 + 2n^2}$

    \item Soma de duas variáveis, comparação de duas variáveis ($1$ unidade
        de custo para cada, multiplicado pelo número de iterações dos laços
        superiores): $\mathbf{2n^3}$

    \item Análogo à linha 15: $\mathbf{2n^3}$

    \item Atribuição a uma variável ($1$ unidade de custo para a atribuição,
        além do número de iterações dos laços superiores, que multiplica o
        resultado final por $n^3$): $\mathbf{n^3}$

\end{enumerate}

$2n + 2 + 2n^2 + 2n + n^2 + n^2 + 2n + 2 + n + 2n + 2 + 2n^2 + 2n + 2n^3
+ 2n^2 + 2n^3 + 2n^3 + n^3$

$T(n) = 7n^3 + 8n^2 + 11n + 6 \rightarrow \mathbf{O(n^3)}$

\end{document}
